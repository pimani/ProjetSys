\documentclass[12pt]{article}
\usepackage[francais]{babel}
\usepackage[utf8]{inputenc}  
\usepackage[T1]{fontenc}  


\title{Projet Systéme d'exploitation}
\author{Pierre Barthélemy, Mathieu Blondel}
\date{\today}

\begin{document}
\maketitle
\newpage
\tableofcontents
\newpage
\section{Introduction}
    Réalisation d'un lanceur de commande en ligne de commande, le projet permétera de lancer des commandes à partir de plusieurs client en déffinissant des pipes comme entrée et sortie standard à ses commande. Les commandes sont éxécuter par un unique serveur.
    Le programme est déveloper en 3 partie, une file synchonisée, un serveur et un client.
\section*{file synchronisée}
    La liste d'attente synchronisée est réaliser à partie d'un espace de mémoire partagé, de deux sémaphore permétant l'accès à plusieurs processus, il ressoit les demandes des clients, se fait lire et vider par le serveur.
\section*{serveur}
    Le serveur va en récupérant la demande venant d'un client via la file synchonisée, éxécuter la commande voulus dans un nouveau procéssus avec les pipes fournis comme entrée et sortie standard.
\section*{client}
    Le client créer deux pipes pour l'entrée et la sortie standard, ajoute sa demande à la file d'attente, écoute et envois ses donnée sur les pipes.

\newpage
\section{File synchronisée}
    La file synchonisée utilise une structure de donnée pour permetre l'accée à ses fonctions:\\
    La structure stocke le nom de l'espace de mémoire partagé et ses droits.
    la liste des fonctions implémenter :
    \begin{description}
        \item [file *file\_vide(void)] Créer une structure correspondant à une file vide avec des éléments de taille
        fixe de taille size.\\
        Renvois NULL en cas de dépassement de capacité mémoire ou si size <= 0\\
        sinon renvois l'addresse d'un descripteur permétant d'y accéder.
        
        \item [const struct info *file\_ouvre(const char *name, int oflag, mode\_t mod)] Ouvre un espace mémoire éxistant et renvois une structure affin d'y accéder avec les autres fonctions de la librairie.\\
        Renvois NULL en cas d'érreur.
        
        \item [const void *file\_ajout(file *f, const void *ptr, size\_t msize, size\_t ns)]  Ajoute l'éléments pointer par ptr à la fin de la file décrit pas f.\\
        Si la file est pleine, tante de doubler la taille de la liste, en cas d'échec
        devient bloquant en attendant que la file se vide.\\
        Renvois NULL si ptr == null ou en cas de dépassement de capacité mémoire, sinon renvois ptr.
        
        \item [const void *file\_retirer(file *f)] Défile la file s, renvois nulle si la file et vide sinon renvois l'address d'une copie de l'élément qui était au début de la file, la libération de la mémoire lier à cette éléments est laisser à la discrétion de l'utisateur.\\
        L'opération est bloquante tant en attendant de pouvoir dépiler un élément.
        
        \item [bool file\_est\_vide(const file *f)] Renvois vrais ou faux si la file s est vide.
        
        \item [int file\_vider(info *f)] libérer l'espace de mémoire partager et détruit les sémaphore, renvois -1 en cas d'érreur
    \end{description}
    La file est implémenter avec une liste flexible dans un espace mémoire partager, la liste peux voir sa taille doubler en cas de besoin pour permetre un plus grand nombre d'éléments. \\
    Le créateur de la file doit définir une taille au éléments étant mis dans la file.\\
    Pour permetre l'accée à plusieurs client, un algorithme producteur consomateur est implémenter avec 3 sémaphore stocker dans l'espace de mémoire partager.

\newpage
\section{Serveur}
    Le serveur lis la file synchonisée pour avoir les différentes commandes à éxécuter, quand une commande est ressus il créer un nouveau thread qui aura pour tache de s'en occuper.\\
    Les demandes sont transmise à traver un structure regroupant diverses informations :
    \begin{description}
        \item [argv] Liste des arguments de la commande, inclus la commande elle même
        \item [envp] Arguments d'environement de la commande.
        \item [tube\in] Nom du tube à relier à l'entrée standard de la commande.
        \item [tube\_out] Nom du tube à relier à la sortie standard de la commande.
    \end{description}
    Les tubes ressus ouvert dans l'ordre :
    \begin{enumerate}
        \item entrée standard permétant au client d'envoyer plus d'information à la commande.
        \item sortie standard permétant au client de recevoir les méssages de la commande.
    \end{enumerate}
    L'ouverture des tubes est à la discrétion du client.\\
    Dans le cas d'une commande transmise invalide, le serveur coupe le tube de retour sans rien écrire dedans.
    Le serveur ne se stope que dans le cas d'une érreur ou d'un signal d'arrét par l'utilisateur.
\newpage
\section{Client}
    Les tubes à l'honneur et le devoir de nommée les tubes qu'il veux utiliser.
\end{document}
