\documentclass[12pt]{article}
\usepackage[francais]{babel}
\usepackage[utf8]{inputenc}  
\usepackage[T1]{fontenc}  

\title{Projet Systéme d'exploitation}
\author{Pierre Barthélemy, Mathieu Blondel}
\date{\today}

\begin{document}
\maketitle
\newpage
\tableofcontents
\newpage
\section{Introduction}
    Réalisation d'un lanceur de commande en ligne de commande, le projet permétera de lancer des commandes à partir de plusieurs client en déffinissant des pipes comme entrée et sortie standard à ses commande. Les commandes sont éxécuter par un unique serveur.
    Le programme est déveloper en 3 partie, une file synchonisée, un serveur et un client.
\section*{file synchronisée}
    La liste d'attente synchronisée est réaliser à partie d'un espace de mémoire partagé, de deux sémaphore permétant l'accès à plusieurs processus, il ressoit les demandes des clients, se fait lire et vider par le serveur.
\section*{serveur}
    Le serveur va en récupérant la demande venant d'un client via la file synchonisée, éxécuter la commande voulus dans un nouveau procéssus avec les pipes fournis comme entrée et sortie standard.
\section*{client}
    Le client créer deux pipes pour l'entrée et la sortie standard, ajoute sa demande à la file d'attente, écoute et envois ses donnée sur les pipes.

\newpage
\section{File synchronisée}
    La file synchonisée utilise une structure de donnée pour permetre l'accée :\\
    La structure stocke le nom de l'espace de mémoire partagé et ses droits.
    la liste des fonctions implémenter :
    \begin{description}
        \item [file *file\_vide(void)] Créer une structure correspondant à une file vide et un espace de mémoire partagé nommé sur le qu'elle se trouve la structure.\\
        Renvois une structure de donnée décrivant la file afin de pouvoir y accéder.\\
        Renvois NULL en cas de dépassement de capacité mémoire.
        \item [const void *file\_ajout(file *f, const void *ptr, size\_t msize, size\_t ns)]  Ajoute l'éléments pointer par ptr à la fin de la file s.\\
        Renvois ptr.\\
        Renvois NULL si ptr == null ou en cas de dépassement de capacité mémoire.;
        \item [const void *file\_retirer(file *f)] Défile la file s, renvois nulle si la file et vide sinon renvois l'address de l'élément qui était au début de la file.
        \item [bool file\_est\_vide(const file *f)] Renvois vrais ou faux si la file s est vide.
    \end{description}

\newpage
\section{Serveur}
    Le serveur lis la file synchonisée pour avoir les différentes commandes à éxécuter, quand une commande est ressus il créer un nouveau thread qui aura pour tache de s'en occuper.

\newpage
\section{Client}

\end{document}
