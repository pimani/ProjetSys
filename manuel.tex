\documentclass[12pt]{article}
\usepackage[francais]{babel}
\usepackage[utf8]{inputenc}
\usepackage[T1]{fontenc}

\title{Manuel d'utilisation du lanceur de commandes}
\author{Pierre Barthélemy & Mathieu Blondel}
\date{\today}

\begin{document}
\maketitle
\newpage
\tableofcontents
\newpage
\section{Introduction}
    Réalisation d'un lanceur de commandes en ligne de commande, le projet permet de lancer des commandes à partir d'un ou plusieurs clients en définissant des noms de tubes comme entrée et sortie standards de ses commandes. Les commandes sont éxécutée par un serveur unique.
    Le programme est développé en 2 sous-programmes, un serveur et un client.
\section*{Serveur}
    C'est le serveur qui récupère les demandes, les traite et fait le lien entre le programme et le client, il doit être lancé en premier.
\section*{Client}
    Le client envoie les demandes de l'utilisateur au serveur et il interagit avec, il doit être lancé après le serveur sinon il ne pourra rien faire et enverra une erreur.
\newpage

\section{Serveur}
    L'utilisateur n'a rien à faire durant l'exécution du serveur, il n'intervient qu'en début et en fin de tâche.
    Pour l'exécuter, l'utilisateur n'a qu'à lancer l'éxécutable, pour l'arréter, envoyer un signal ou une interruption clavier.
    Le serveur doit toujours être lancé en premier car il s'agit de lui qui crée l'espace de communication entre les processus.
    Si il le souhaite, l'utilisateur pourrait rediriger la sortie standard du programme afin d'avoir les logs d'écris dans un fichier.
\newpage

\section{Client}
    Le client peut être exécuté à n'importe quel moment de la vie du serveur.
    Son utilisation se fait dans le terminal via des entrées clavier :
      \begin{enumerate}
        \item Lancement du logiciel
        \item Donne le premier argument de la commande qui est la commande elle-même.
        \item Donne jusqu'à x autre aguments, x étant un entier positif non nul défini à la compilation.
        \item Donne si il le souhaite des variables d'environnements.
      \end{enumerate}
    À la suite de cela, la demande est envoyée. Quand elle sera traitée, l'utilisateur aura la main
    sur l'entrée et la sortie standards de sa commandes. Si il veux mettre fin à l'envoi de données, il peut à tout moment envoyer un signal de fin de fichier avec ctrl+d.

\end{document}
